\documentclass{article}


\usepackage[utf8]{inputenc}
\usepackage[T2B]{fontenc}
\usepackage[english,russian]{babel}

\title{\textbf{ЛР 2}} \author{Недосеков Иван 206}

\usepackage{sagetex}
\setlength{\sagetexindent}{10ex}

\usepackage{geometry}
\geometry{a4paper, total={170mm,257mm}, left=20mm, right=20mm, top=20mm, bottom=20mm}

\usepackage{xcolor}
\usepackage{titlesec}

\definecolor{myred1}{RGB}{107, 14, 14}
\definecolor{mygreen1}{RGB}{0, 102, 0}

\begin{document}

\maketitle
\titleformat{\section}[block]{\color{red}\Large\bfseries\filcenter}{}{1em}{}

\section{3. Упростить уравнение поверхности второго порядка в простраснтвe}

\textbf{Вариант 17:}
\begin{sageblock}
var('x y z')
f(x, y, z) = 8*x^2 - 2*x*y - 4*y^2 + 2*x*z - 2*y*z + 3*z^2 + 7*x + 8*y + 9*z - 10
\end{sageblock}
$f(x, y, z) = \sage{f(x, y, z)}$\\
\textbf{Составим матрицу A, столбец а и свободный член а0} 
\begin{sageblock}
A = matrix([
            ( 8, -1,   1),
            (-1, -4,  -1), 
            ( 1, -1,   3) 
           ])
a = matrix([
            [7],
            [8],
            [9]
            ])
a0 = -10
\end{sageblock}

\textbf{Константы для отрисовки и работы} 
\begin{sageblock}
xmin, xmax = -50, 50
ymin, ymax = -50, 50
zmin, zmax = -50, 50
\end{sageblock}
\textbf{График заданной поверхности}
\begin{sageblock}
p = implicit_plot3d(f(x=x,y=y,z=z),(x, xmin, xmax), (y, ymin, ymax), (z, zmin, zmax))
\end{sageblock}
\begin{center}
\sageplot{p}
\end{center}
\textbf{Находим собственные векторы и формируем из них матрицу перехода.}
\begin{sageblock}
def norm(v):
    n = 0
    for i in range(3):
        n += v[i]*v[i]
    n = sqrt(n)
    for i in range(3):
        v[i] /= n
    return v

ev = A.eigenvalues()
vectors = A.eigenvectors_right()
v1 = norm(vectors[0][1][0])
v2 = norm(vectors[1][1][0])
v3 = norm(vectors[2][1][0])

ST = identity_matrix(RR, 3)
norm_vectors = [v1, v2, v3]
for i in range(3):
    for j in range(3):
        ST[i,j] = norm_vectors[i][j]

a_ = ST * a
\end{sageblock}

\textbf{Составляем поверхность в новом виде}
\begin{sageblock}
u(x, y, z) = *(a_[0]*x + a_[1]*y + a_[2]*z) # освобождаем от кортежа
kf(x, y, z) = ev[0]*x^2 + ev[1]*y^2 + ev[2]*z^2 + u + a0
p = implicit_plot3d(kf(x=x, y=y, z=z), (x, xmin, xmax), (y, ymin, ymax), (z, zmin, zmax))
\end{sageblock}
\begin{center}
\sageplot{p}
\end{center}


\end{document}